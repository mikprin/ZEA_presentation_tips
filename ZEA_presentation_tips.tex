% "Станет проще"

\documentclass[a4paper,12pt]{article} % тип документа

% report, book

%  Русский язык

\usepackage[T2A]{fontenc}			% кодировка
\usepackage[utf8]{inputenc}			% кодировка исходного текста
\usepackage{graphicx}
\usepackage[english,russian]{babel}	% локализация и переносы


%отступ
\usepackage[left=2cm,right=1cm,
    top=2cm,bottom=2cm,bindingoffset=0cm]{geometry}

% Математика
\usepackage{amsmath,amsfonts,amssymb,amsthm,mathtools} 
\usepackage{csvsimple}
\usepackage{multirow}

\usepackage{hyperref}
\usepackage{wasysym}
\usepackage{subcaption}
\usepackage{verbatim}
\usepackage{hyperref}
\usepackage{float}
\usepackage{enumerate}
\usepackage[dvipsnames]{xcolor}
\usepackage{rotating}
\usepackage{textcomp}

%Заговолок
%\graphicspath{ {images/} }


\begin{titlepage}
\author{АВТОР }
\title{НАЗВАНИЕ}
\date{\today}
\end{titlepage}



\begin{document} % начало документа
%\maketitle

\section{Presentation}
\paragraph{Introduction}
Thank you for chance to dig in this field.

\paragraph{Problem of scaling}


\paragraph{Limited bonding area and heat load}
\paragraph{Input signal specifications} Exitation
\paragraph{Chip interconnect}

\paragraph{System overview} Digital. Memory. Bias DAC. RF DAC.
\paragraph{Memory owerview}
Advantages and disadvantages!!!
\paragraph{Clock control}
\paragraph{Why not SRAM} + SRAM diagram
\paragraph{Bias generation}

\paragraph{DAC design}
\subparagraph{Capacitive DAC}
\subparagraph{DAC considerations} Noise: Resistor=R+noise source  $ \overline{v_n^2} = 4k_BTR_{out}B  $ \hspace{2mm}
$R_{out\_Kelvin}=2^{n-2}R_u$ \\
Sample and hold: $\overline{v_n^2} = k_BT/C_{out}$


The bandwidth is set in accordance with the filtering behaviour of the qubit.
%\newpage


\paragraph{DAC pictures}
Area consumption of the different components of the
control electronics (labeled as in Fig. 3) for different values of
the resolution of the bias signal (a) and of the RF signal (b).
In panel a) the RF resolution is assumed to be 10 bits, while
in panel b) the bias resolution is fixed to 12 bits.
Comparison of DACs with component dimensions
 $R_{u,Ladder} = 150R$
a) Area consumption of the different DAC architectures for
different resolutions, independent of operating conditions b)
Power consumption of the DAC architecture, under the op-
erating condition of the bias generation unit ($V_{range} = 1 V$
and $f_{refresh} = 1.11 MHz$). c) Same as in b), but now con-
sidering the operating conditions of the RF generation unit
($V_{range} = 4 mV$ and $f_{refresh} = = 300 MHz$).

\paragraph{Switch design}

\paragraph{power consumption}

\paragraph{low voltage transistors}

\paragraph{new process}

\paragraph{isolation}

\paragraph{tranch caps}

\paragraph{summary} Я не парился над оформлением ибо эт ваша статья!


\paragraph{Similarities with my previous IC design experience}
\begin{itemize}
    \item Aim at precision circuit design like in measuring tech
    \item Restricted environment
\end{itemize}

\section{Questions}

\end{document}